\documentclass[a4paper]{article}
	\usepackage[colorlinks, linkcolor=red]{hyperref}
	\usepackage{ctex}
	\IfFontExistsTF{Times New Roman}{
		\setmainfont{Times New Roman}
	}{
		\usepackage{newtxtext,newtxmath}
	}
	
	%设置文档中文字体。优先次序:中易 > Adobe > 华文(Mac) > Fandol
	\IfFontExistsTF{SimSun}{
		\setCJKmainfont[AutoFakeBold=2,ItalicFont=KaiTi]{SimSun}
	}{
		\IfFontExistsTF{AdobeSongStd-Light}{
			\setCJKmainfont[AutoFakeBold=2,ItalicFont=AdobeKaitiStd-Regular]{AdobeSongStd-Light}
		}{
			\IfFontExistsTF{STSong}{
				\setCJKmainfont[AutoFakeBold=2,BoldFont=STHeiti,ItalicFont=STKaiti]{STSong}
			}{
				\setCJKmainfont[AutoFakeBold=2,ItalicFont=FandolKai-Regular]{FandolSong-Regular}
			}
		}
	}
	\IfFontExistsTF{SimHei}{
		\setCJKsansfont[AutoFakeBold=2]{SimHei}
	}{
		\IfFontExistsTF{AdobeHeitiStd-Regular}{
			\setCJKsansfont[AutoFakeBold=2]{AdobeHeitiStd-Regular}
		}{
			\IfFontExistsTF{STHeiti}{
				\setCJKsansfont [AutoFakeBold=2]{STHeiti}
			}{
				\setCJKsansfont[AutoFakeBold=2]{FandolHei-Regular}
			}
		}
	}
	\usepackage[
	restoremathleading=false,
	UseMSWordMultipleLineSpacing,
	MSWordLineSpacingMultiple=1.05
	]{zhlineskip}
	\title{刘子汉 (Altair)}
	\author{\url{altair.liu@sjtu.edu.cn},+86 159 0215 7531}
	\date{个人主页:\url{http://subjectnoi.github.io/about/}}
	\usepackage{graphicx}
	\usepackage{booktabs}
	\usepackage{amsmath}
	\usepackage{listings}
	\usepackage{geometry}
	\geometry{left=1cm, right=1cm, top=0cm, bottom=0cm}
	\begin{document}
		\maketitle
		\section*{教育}
		\rule[1pt]{18cm}{0.01em}\\
		\textbf{2015.09-2019.06} \hfill \textbf{学士,计算机科学与技术系,华东师范大学}
		\begin{itemize}
		\item 平均绩点:3.81/4.00,排名:5/116。
		\item 2018、2019年获得奖学金,2018年获得美国大学生数学建模竞赛S奖,2016年获得CCCC全国大学生程序设计竞赛三等奖。
		\item 获得华东师范大学2019年免试研究生资格,并被上海交通大学录取。
		\end{itemize}
		\textbf{2019.09-Now} \hfill \textbf{硕士,计算机科学与技术系,上海交通大学}
		\begin{itemize}
		\item 进行计算机系统组成架构、并行计算、编译器等方面的工作。
		\end{itemize}
		\section*{工作}
		\rule[1pt]{18cm}{0.01em}\\
		\textbf{2018.08-2019.01} \hfill \textbf{云应用开发,IBSO,SAP(实习)}
		\begin{itemize}
		\item 从事基于Java以及相关工具(Spring,OData,MongoDB等)的部署于Cloud Foundry的商用工具(MaCo Cloud)的开发工作,以及协助使用Jenkins与git进行版本管理,项目使用Maven构建。
		\item 从事基于Java,SAPUI5以及相关工具的部署于Cloud Foundry的S/4 HANA应用的开发工作,使用Karma,QUnit与Opa5进行相应单元测试、集成测试。
		\end{itemize}
		\textbf{2019.01-2019.06} \hfill \textbf{GPU流多处理器架构,ARCH,NVIDIA(实习)}
		\begin{itemize}
		\item 从事下一代硬件(Ampere与Hopper架构)的软件层面功能与性能仿真、验证工具的开发。项目基于C/C++/CUDA、PTX中间代码、SASS汇编代码与少量Verilog HDL硬件代码。主要进行新指令UMMA的功能实现、调试以及对软件仿真与片上仿真的结果进行对比、研究、修改软件仿真代码。项目使用cmake构建、Perforce进行版本控制。
		\end{itemize}
		\section*{项目经验}
		\rule[1pt]{18cm}{0.01em}
		\begin{itemize}
		\item 机器学习相关课程项目,包括基于DCGAN的图像合成,基于CUDA、OpenMP、MPI的并行化算法如DCGAN,基于$ \alpha-\beta $剪枝的最大最小搜索的五子棋,基于NSGA II的规划任务等。
		\item 从硬件到软件层面再机器学习应用方面对图灵架构GPU进行的研究,作为毕业设计。
		\item 简化的类C语言编译器,使用Lex,Yacc以及LLVM。 
		\item 基于虚幻4引擎开发的第三人称视角动作游戏。
		\end{itemize}
		\section*{技能}
		\rule[1pt]{18cm}{0.01em}\\
		C/C++/CUDA,ACM/机器学习/深度学习相关算法及数据结构,Linux,\LaTeX,Python,Java,SQL/MongoDB,虚幻4引擎,Verilog HDL.
		
	\end{document}


